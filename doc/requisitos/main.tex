\documentclass{article}
\usepackage{fullpage}

%load needed packages
\usepackage{graphicx}
\usepackage{array}
\usepackage{booktabs}
\usepackage[utf8]{inputenc}
\usepackage[T1]{fontenc}

\usepackage[spanish]{babel} % Paquete para el idioma español
\usepackage{float}  % Necesario para [H]
\usepackage{listings}
\usepackage{xcolor}

\definecolor{codegreen}{HTML}{5AB2FF}
\definecolor{morado}{HTML}{AD88C6}
\definecolor{BG}{HTML}{EEEEEE}
\definecolor{azul}{HTML}{4D869C}
\definecolor{sqlblue}{HTML}{FF8C00} % Color para las palabras clave SQL

% Estilo para DDL
\lstdefinestyle{ddlstyle}{
	language=SQL,
	backgroundcolor=\color{BG},
	commentstyle=\color{codegreen},
	basicstyle=\ttfamily\small,
	keywordstyle=\color{azul},
	stringstyle=\color{morado},
	showstringspaces=false,
	breaklines=true,
	frame=shadowbox,
	numbers=left,
	numberstyle=\tiny\color{gray},
	captionpos=b,
}

% Estilo para SQL
\lstdefinestyle{sqlstyle}{
	language=SQL,
	backgroundcolor=\color{BG},
	commentstyle=\color{codegreen},
	basicstyle=\ttfamily\small,
	keywordstyle=\color{sqlblue}, % Color diferente para palabras clave SQL
	stringstyle=\color{morado},
	showstringspaces=false,
	breaklines=true,
	frame=shadowbox,
	numbers=left,
	numberstyle=\tiny\color{gray},
	captionpos=b,
}

\begin{document}
	
	% Portada
	\begin{titlepage}
		\centering
		\vspace*{3cm}
		
		% Título destacado
		{\Huge \textbf{Plataforma web de rehabilitación a distancia}\\[0.5cm]}
		
		% Espacio y logotipo (si lo tienes, por ejemplo el logo de tu universidad)
		\vspace{2cm}
		\includegraphics[width=0.3\textwidth]{images/uma_logo.jpg}\\[1cm]
		
		% Nombre del autor
		{\LARGE \textbf{Organización basada en componentes}\\[0.5cm]}
		{\large \textit{Ingeniería web}\\
			Universidad de Málaga\\
		}
		
		\vfill
		
		% Fecha en la parte inferior de la página
		{\large Septiembre 2024}
	\end{titlepage}
	
	% índice
	\tableofcontents
	
	\newpage
	
	\section{Introducción}
	La rehabilitación es una fase crítica en el proceso de recuperación de pacientes que han
	sufrido lesiones, intervenciones quirúrgicas o padecen de enfermedades crónicas.
	Tradicionalmente, la rehabilitación se realiza de manera presencial, lo que puede generar
	barreras logísticas, económicas y geográficas tanto para los pacientes como para los
	profesionales de la salud. Ante esta realidad, surge la necesidad de una Plataforma Web de
	Rehabilitación a Distancia, cuyo objetivo es facilitar el acceso a programas de rehabilitación
	personalizados, ofrecer seguimiento remoto y mejorar la calidad de vida de los pacientes sin
	la necesidad de visitas constantes a centros de rehabilitación.
	Este proyecto plantea el desarrollo de una plataforma web integral que permita a los
	pacientes recibir tratamientos de rehabilitación de manera remota, mientras que los
	profesionales de la salud pueden monitorizar el progreso y ajustar las terapias en tiempo real.
	A través de la combinación de herramientas tecnológicas avanzadas como la videollamada,
	la telemetría de dispositivos médicos, y la inteligencia artificial para la personalización de
	ejercicios, esta plataforma buscará convertirse en un referente en el ámbito de la telerehabilitación.
	
	\section{Descripción de Stakeholders}
	
	\begin{table}[ht]
		\centering
		\caption{Stakeholders Principales}
		\begin{tabular}{@{} p{3cm} p{6cm} p{7cm} @{}}
			\toprule
			\textbf{Nombre} & \textbf{Representa} & \textbf{Rol} \\
			\midrule
			Profesional médico & 
			Especialista de la salud encargado del seguimiento y control de la rehabilitación a distancia & 
			Supervisa y evalúa el progreso de los pacientes. \newline Personaliza los planes de tratamiento, proporciona feedback y ajusta terapias en tiempo real. \\
			
			\addlinespace
			
			Paciente & 
			U & 
			R. \\
			
			\addlinespace
			
			Desarrollador de Software & 
			Equipo encargado del desarrollo del software de la aplicación o página web & 
			Desarrolla y mantiene la funcionalidad del sistema, asegurándose de que los datos de los pacientes se almacenen y se procesen correctamente. \\
			\bottomrule
		\end{tabular}
	\end{table}
	
	\section{Requisitos}
	
	\subsection{Funcionales}
	
\newcounter{requisitosFuncionales}
\begin{table}[H]
	\centering
	\caption{Requisitos Funcionales}
	\begin{tabular}{@{} p{2.5cm} p{6.5cm} p{3cm} p{3cm} @{}}
		\toprule
		\textbf{ID} & \textbf{Descripción} & \textbf{Obligatoriedad} & \textbf{Padres} \\
		\midrule
		\addlinespace
		\stepcounter{requisitosFuncionales} RF\therequisitosFuncionales & Los pacientes podrán registrar sus resultados y cumplimiento diario del plan de ejercicios, especificando series, repeticiones y niveles de dolor. & Obligatorio & - \\
		\addlinespace
		\stepcounter{requisitosFuncionales} RF\therequisitosFuncionales & La plataforma recopilará datos de dispositivos wearables como frecuencia cardíaca, calorías quemadas y tiempo de actividad. & Obligatorio & - \\
		\addlinespace
		\stepcounter{requisitosFuncionales} RF\therequisitosFuncionales & Si se detecta un patrón inusual (dolor intenso o incumplimiento de ejercicio), se enviará una alerta al profesional asignado. & Obligatorio & - \\
		\addlinespace
		\stepcounter{requisitosFuncionales} RF\therequisitosFuncionales & Los pacientes podrán agendar consultas a través de videollamadas con su fisioterapeuta. & Obligatorio & - \\
		\addlinespace
		\stepcounter{requisitosFuncionales} RF\therequisitosFuncionales & El sistema de mensajería permitirá la comunicación directa entre paciente y profesional. & Obligatorio & - \\
		\bottomrule
	\end{tabular}
\end{table}

\subsection{No funcionales}

\newcounter{requisitosNoFuncionales}
\begin{table}[H]
	\centering
	\caption{Requisitos No Funcionales}
	\begin{tabular}{@{} p{2.5cm} p{6.5cm} p{3cm} @{}}
		\toprule
		\textbf{ID} & \textbf{Descripción} & \textbf{Obligatoriedad} \\
		\midrule
		\addlinespace
		\stepcounter{requisitosNoFuncionales} RNF\therequisitosNoFuncionales & La plataforma deberá ser accesible desde dispositivos móviles, tabletas y PC. & Obligatorio \\
		\addlinespace
			\stepcounter{requisitosNoFuncionales} RNF\therequisitosNoFuncionales & Todos los datos personales y médicos estarán protegidos mediante cifrado. & Obligatorio \\
	
		\bottomrule
	\end{tabular}
\end{table}
	
	\section{Acceso al Repositorio}
	
	Toda la información adicional, incluyendo el código fuente y la documentación completa de este proyecto, está disponible en el repositorio de GitHub.
	
	% Incluir la bibliografía
	\bibliographystyle{plain}  % Estilo de la bibliografía (por ejemplo, plain, alpha, ieee, etc.)
	\bibliography{bibli}  % Nombre del archivo .bib sin la extensión
	
\end{document}
