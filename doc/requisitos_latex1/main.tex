\documentclass{article}
\usepackage{fullpage}

%load needed packages
\usepackage{graphicx}
\usepackage{array}
\usepackage{booktabs}
\usepackage[utf8]{inputenc}
\usepackage[T1]{fontenc}
\usepackage{hyperref}

\usepackage[spanish]{babel} % Paquete para el idioma español
\usepackage{float}  % Necesario para [H]
\usepackage{listings}
\usepackage{xcolor}
\usepackage{longtable} 

\definecolor{codegreen}{HTML}{5AB2FF}
\definecolor{morado}{HTML}{AD88C6}
\definecolor{BG}{HTML}{EEEEEE}
\definecolor{azul}{HTML}{4D869C}
\definecolor{sqlblue}{HTML}{FF8C00} % Color para las palabras clave SQL

% Estilo para DDL
\lstdefinestyle{ddlstyle}{
	language=SQL,
	backgroundcolor=\color{BG},
	commentstyle=\color{codegreen},
	basicstyle=\ttfamily\small,
	keywordstyle=\color{azul},
	stringstyle=\color{morado},
	showstringspaces=false,
	breaklines=true,
	frame=shadowbox,
	numbers=left,
	numberstyle=\tiny\color{gray},
	captionpos=b,
}

% Estilo para SQL
\lstdefinestyle{sqlstyle}{
	language=SQL,
	backgroundcolor=\color{BG},
	commentstyle=\color{codegreen},
	basicstyle=\ttfamily\small,
	keywordstyle=\color{sqlblue}, % Color diferente para palabras clave SQL
	stringstyle=\color{morado},
	showstringspaces=false,
	breaklines=true,
	frame=shadowbox,
	numbers=left,
	numberstyle=\tiny\color{gray},
	captionpos=b,
}

\begin{document}
	
	% Portada
	\begin{titlepage}
		\centering
		\vspace*{3cm}
		
		% Título destacado
		{\Huge \textbf{Plataforma Web de Rehabilitación a Distancia}\\[0.5cm]}
		
		{\Large \textbf{Entrega Requisitos: Draft 2}\\[0.5cm]}
		
		% Espacio y logotipo (si lo tienes, por ejemplo el logo de tu universidad)
		\vspace{2cm}
		\includegraphics[width=0.3\textwidth]{images/uma_logo.jpg}\\[1cm]
		
		% Nombre del autor
		{\LARGE \textbf{Organización basada en componentes}\\[0.5cm]}
		{\large \textit{Integrantes:}\\
			Cuevas Rodríguez, Marta\\
			de Pablo, Diego\\
			Silva Rodríguez, Alejandro\\
			Soriano Muñoz, Juan Ignacio\\
		}
		\vfill
		{\large \textit{Ingeniería web}\\
			Universidad de Málaga\\
		}
		
		\vfill
		
		% Fecha en la parte inferior de la página
		{\large Octubre 2024}
	\end{titlepage}
	
	% índice
	\tableofcontents
	
	\newpage
	
	\section{Introducción}

La rehabilitación es una fase crítica en el proceso de recuperación de pacientes que han sufrido lesiones, intervenciones quirúrgicas o padecen enfermedades crónicas. Tradicionalmente, la rehabilitación se realiza de manera presencial, lo que puede generar barreras logísticas, económicas y geográficas tanto para los pacientes como para los profesionales de la salud. Ante esta realidad, surge la necesidad de una Plataforma Web de Rehabilitación a Distancia, cuyo objetivo es facilitar el acceso a programas de rehabilitación personalizados, ofrecer seguimiento remoto y mejorar la calidad de vida de los pacientes sin la necesidad de visitas constantes a centros de rehabilitación.
\\
\\
Este proyecto plantea el desarrollo de una plataforma web integral que permita a los pacientes recibir tratamientos de rehabilitación de manera remota, mientras que los profesionales de la salud pueden monitorizar el progreso y ajustar las terapias en tiempo real. Los principales stakeholders involucrados en este proyecto incluyen a pacientes, profesionales de la salud (fisioterapeutas, médicos rehabilitadores) y desarrolladores de software. Los pacientes se beneficiarán de un acceso más flexible a sus tratamientos, mientras que los profesionales podrán optimizar el seguimiento clínico y ajustar terapias de manera eficiente.
\\
\\
Entre los posibles casos de uso se encuentran situaciones como la rehabilitación de un paciente con una lesión muscular, que puede realizar sus ejercicios desde casa bajo la supervisión de un fisioterapeuta a través de videollamadas, o un paciente crónico que, mediante dispositivos de telemetría y un registro de ejercicios, permite que su progreso sea monitorizado de forma continua.


\section{Objetivos del Proyecto}
El objetivo principal del proyecto es crear una plataforma web que permita a los pacientes acceder a su proceso de rehabilitación de manera eficiente, a través de programas de tratamiento personalizados que fomenten una mayor adherencia al mismo. La plataforma facilitará la rehabilitación de forma remota, eliminando barreras geográficas y temporales, y creando un entorno flexible donde los pacientes puedan realizar sus ejercicios desde cualquier lugar y en cualquier momento, siempre bajo la supervisión de un profesional de la salud.
\\

Además, la plataforma ofrecerá a los profesionales de la salud herramientas avanzadas para monitorear los logros y evaluar la efectividad de los tratamientos en tiempo real, integrando dispositivos conectados como wearables, videoconferencias y sistemas de mensajería instantánea. Esto permitirá una personalización continua de los planes de rehabilitación, ajustándolos según la evolución y las necesidades específicas de cada paciente, mejorando tanto el seguimiento como la comunicación entre pacientes y profesionales.
\\

Por último, la seguridad y la confidencialidad de los datos serán aspectos fundamentales en el desarrollo de la plataforma. Todos los datos personales y médicos de los pacientes estarán protegidos de acuerdo con los estándares más rigurosos, cumpliendo plenamente con las normativas de privacidad como el RGPD. La plataforma no solo garantizará la protección de estos datos sensibles, sino que también ofrecerá una interfaz accesible y fácil de usar desde cualquier dispositivo, ya sean móviles, tabletas o PCs, asegurando que tanto los pacientes como los profesionales puedan acceder al sistema de manera sencilla y segura desde cualquier lugar.
	
	
	\section{Análisis de Riesgos}
	
	Al implementar un proyecto, siempre existen riesgos que deben ser identificados y gestionados adecuadamente. Es fundamental anticipar los posibles desafíos que puedan surgir durante el desarrollo y la implementación de la plataforma, con el fin de preparar planes de acción que permitan mitigar su impacto. A continuación, se presenta un análisis de riesgos potenciales, clasificándolos según su probabilidad de ocurrencia, impacto y las medidas preventivas o correctivas para su evitación o mitigación.
	
	\begin{table}[H]
	\centering
	\caption{Análisis de Riesgos del Proyecto}
	\begin{tabular}{@{} p{1cm} p{5.5cm} p{3cm} p{5.5cm} @{}}
		\toprule
		\textbf{Id} & \textbf{Descripción} & \textbf{Probabilidad e Impacto} & \textbf{Evitación/Mitigación} \\
		\midrule
		
		R1 & 
		El aumento rápido de usuarios tras la implementación podría saturar los servidores, causando su caída & 
		Probabilidad: Bajo \newline Impacto: Alto & 
		se optimizará la infraestructura y se implementará monitoreo en tiempo real para resolver problemas rápidamente, garantizando la estabilidad del sistema. \\
		
		\addlinespace
		
		R2 & 
		Las personas no logren adaptarse a usar la página. & 
		Probabilidad: Media \newline Impacto: Bajo & 
		Se buscará crear una intefaz integral que pueda ser usada fácilmente sin ningún conocimiento en computación, además de implementar gamificación \\
		
		\addlinespace
		
		R3 & 
		Falta de personal de la salud dispuesto a colaborar en el proyecto & 
		Probabilidad: Bajo \newline Impacto: Bajo & 
		Se considera poco probable que falte personal, ya que su vocación es ayudar a los pacientes. Si esto ocurriera se podría considerar la implementación de un chatbot \\
		
		\addlinespace
		
		R4 & 
		Retrasos en la entrega de programas & 
		Probabilidad: Bajo \newline Impacto: Bajo & 
		Se dará un plazo estimado de entrega para evitar alertar a los usuarios. En caso de retraso, los usuarios podrán contactar con los profesionales para resolver dudas sin depender completamente del programa. \\
		
		\bottomrule
	\end{tabular}
	\end{table}
	
	
	
	
	
	\section{Descripción de Stakeholders}
	
	Un stakeholder (o parte interesada) es cualquier persona, grupo u organización que tiene un interés o impacto directo en un proyecto, producto o empresa, o que se ve afectada por los resultados de dicho proyecto. 
	
	\begin{table}[ht]
		\centering
		\caption{Stakeholders Principales}
		\begin{tabular}{@{} p{3cm} p{6cm} p{7cm} @{}}
			\toprule
			\textbf{Nombre} & \textbf{Representa} & \textbf{Rol} \\
			\midrule
			Profesional médico & 
			Especialista de la salud encargado del seguimiento y control de la rehabilitación a distancia & 
			Supervisa y evalúa el progreso de los pacientes. \newline Personaliza los planes de tratamiento, proporciona feedback y ajusta terapias en tiempo real. \\
			
			\addlinespace
			
			Paciente & 
			Usuarios finales que necesitan realizar terapias de rehabilitación a distancia. & 
			Participan activamente usando la plataforma para seguir sus planes de rehabilitación, registrar su progreso y recibir retroalimentación. Su satisfacción es esencial, ya que su experiencia definirá el éxito del proyecto. \\
			
			\addlinespace
			
			Desarrollador  Software & 
			Equipo encargado del desarrollo del software de la aplicación o página web & 
			Desarrolla y mantiene la funcionalidad del sistema, asegurándose de que los datos de los pacientes se almacenen y se procesen correctamente. \\
			\bottomrule
		\end{tabular}
	\end{table}
	
	

	
	
	\section{Requisitos}
	
	Un requisito es una condición o capacidad que debe cumplirse o tener un sistema, producto o proyecto para satisfacer las necesidades y expectativas de los usuarios o stakeholders, se pueden dividir en funcionales y no funcionales.
	
	\subsection{Funcionales}
	
	Un requisito funcional especifica las capacidades y servicios que debe ofrecer un sistema, definiendo lo que el sistema debe hacer para cumplir con las necesidades del usuario. Por ejemplo, en una aplicación bancaria, un requisito funcional podría ser: "El sistema debe permitir transferir dinero". Estos requisitos suelen describirse a través de escenarios de casos de uso y especificaciones formateadas.
	
	
	Para este proyecto se creará una tabla para describir los requisitos, cada columna corresponderá a lo siguiente:
	
	
	\begin{itemize}
		\item \textbf{ID}: Identificador único del requisito funcional, que facilita su referencia y seguimiento. Ejemplo: RF0.
		
		\item \textbf{Descripción}: Breve enunciado que detalla lo que el requisito funcional implica, incluyendo las acciones que el usuario podrá realizar. Ejemplo: "Los usuarios podrán registrarse en la plataforma según su rol."
		
		\item \textbf{Obligatoriedad}: Indica si el requisito es obligatorio o opcional para el sistema, lo que ayuda a priorizar su implementación. 
		
		\item \textbf{Dependencia}: Enumera los requisitos de los que depende este requisito, lo que ayuda a entender la secuencia y relaciones entre los requisitos.
		
		\item \textbf{Trazabilidad}: Proporciona información sobre la característica de la plataforma o documento donde se relaciona este requisito, permitiendo seguir el origen y contexto del requisito. Ejemplo: "Características de la Plataforma, 1.", se refiere al documento de requisitos que en la página 3 tiene una sección de Características de la Plataforma 
	\end{itemize}
	

	
\newcounter{requisitosFuncionales}
\begin{longtable}{@{} p{1.5cm} p{5cm} p{3cm} p{2cm} p{3cm} @{}}
	\caption{Requisitos Funcionales}\\
	\toprule
	\textbf{ID} & \textbf{Descripción} & \textbf{Obligatoriedad} & \textbf{Dependencia} & \textbf{Trazabilidad} \\
	\midrule
	\endfirsthead
	
	\toprule
	\textbf{ID} & \textbf{Descripción} & \textbf{Obligatoriedad} & \textbf{Dependencia} & \textbf{Trazabilidad} \\
	\midrule
	\endhead
	
	\addlinespace 
	RF\therequisitosFuncionales & Los usuarios podrán registrarse en la plataforma según su rol. & Obligatorio & Ninguna & Características de la Plataforma, 1  \\
	\addlinespace \stepcounter{requisitosFuncionales}
	RF\therequisitosFuncionales & Los usuarios iniciarán sesión en la plataforma con acceso según su rol. & Obligatorio & RF0 & Características de la Plataforma, 1 \\  
	\addlinespace \stepcounter{requisitosFuncionales}
	RF\therequisitosFuncionales & Los pacientes podrán acceder a su plan de rehabilitación. & Obligatorio & RF1, RF9 & Características de la Plataforma, 1 \\ 
	\addlinespace \stepcounter{requisitosFuncionales}
	RF\therequisitosFuncionales & Los pacientes podrán visualizar su historial de ejercicios. & Obligatorio & RF2 & Características de la Plataforma, 1 \\ 
	\addlinespace \stepcounter{requisitosFuncionales}
	RF\therequisitosFuncionales & Los pacientes recibirán recordatorios de las sesiones. & Opcional & RF2 & Características de la Plataforma, 1 \\
	\addlinespace \stepcounter{requisitosFuncionales}
	RF\therequisitosFuncionales & Los pacientes podrán registrar sus resultados y cumplimiento diario del plan de ejercicios. & Obligatorio & Ninguna & Características de la Plataforma, 3 \\ 
	\addlinespace \stepcounter{requisitosFuncionales}
	RF\therequisitosFuncionales & Los pacientes podrán agendar consultas a través de videollamadas con médico. & Obligatorio & Ninguna & Características de la Plataforma, 4 \\
	\addlinespace \stepcounter{requisitosFuncionales}
	RF\therequisitosFuncionales & Los profesionales podrán acceder a la información de sus pacientes. & Obligatorio & RF1 & Características de la Plataforma, 1 \\ 
	\addlinespace \stepcounter{requisitosFuncionales}
	RF\therequisitosFuncionales & Los profesionales podrán programar sesiones de seguimiento con sus pacientes. & Obligatorio & RF7 & Características de la Plataforma,  \\ 
	\addlinespace \stepcounter{requisitosFuncionales}
	RF\therequisitosFuncionales & Los profesionales podrán crear planes de rehabilitación personalizados. & Obligatorio & RF7 & Características de la Plataforma, 2  \\ 
	\addlinespace \stepcounter{requisitosFuncionales}
	RF\therequisitosFuncionales & Los profesionales podrán actualizar los planes de rehabilitación de los pacientes. & Obligatorio & RF9 & Características de la Plataforma, 2 \\ 
	\addlinespace \stepcounter{requisitosFuncionales}
	RF\therequisitosFuncionales & La plataforma proporcionará una base de datos con plantillas de ejercicios. & Obligatorio & Ninguna & Características de la Plataforma,  \\ 

	\addlinespace \stepcounter{requisitosFuncionales}
	RF\therequisitosFuncionales & La plataforma recopilará datos de dispositivos wearables. & Obligatorio & Ninguna & Características de la Plataforma, 3 \\
	\addlinespace \stepcounter{requisitosFuncionales}
	RF\therequisitosFuncionales & Se enviará una alerta al profesional en caso de detección de un patrón inusual. & Obligatorio & RF14 & Características de la Plataforma, 3 \\
	\addlinespace \stepcounter{requisitosFuncionales}
	RF\therequisitosFuncionales & El sistema de mensajería permitirá la comunicación directa entre paciente y profesional. & Obligatorio & RF6 & Características de la Plataforma, 4 \\
	\addlinespace \stepcounter{requisitosFuncionales}
	
	RF\therequisitosFuncionales & Los profesionales podrán acceder a paneles interactivos del progreso del paciente. & Obligatorio & RF5 & Características de la Plataforma, 7  \\
	\addlinespace
	\stepcounter{requisitosFuncionales}RF\therequisitosFuncionales & Los pacientes pueden desbloquear logros (gamificación) al cumplir con sus ejercicios diarios. & Opcional & RF5 & Charla con Paciente (chatGPT) \\
	\addlinespace
	\bottomrule
\end{longtable}



\subsection{No funcionales}

Un requisito no funcional describe las propiedades y características de las capacidades del sistema, así como el nivel de servicio deseado. Por ejemplo, un requisito no funcional podría ser: "La aplicación web debe soportar el uso de 2500 usuarios de forma concurrente". A diferencia de los requisitos funcionales, que se centran en qué debe hacer el sistema, los requisitos no funcionales abordan cómo debe comportarse, incluyendo restricciones del proyecto y propiedades de las interfaces del sistema.

\newcounter{requisitosNoFuncionales}
\begin{table}[H]
	\centering
	\caption{Requisitos No Funcionales}
	\begin{tabular}{@{} p{2.5cm} p{6.5cm} p{3cm} p{3cm} @{}}
		\toprule
		\textbf{ID} & \textbf{Descripción} & \textbf{Obligatoriedad} & \textbf{Trazabilidad} \\
		\midrule
		\addlinespace
		\stepcounter{requisitosNoFuncionales} RNF\therequisitosNoFuncionales & La plataforma deberá ser accesible desde dispositivos móviles, tabletas y PC. & Obligatorio & Características de la Plataforma, 1 \\
		\addlinespace
		\stepcounter{requisitosNoFuncionales} RNF\therequisitosNoFuncionales & Se implementará un sistema de autenticación seguro para proteger los datos personales y médicos de los usuarios. & Obligatorio & Características de la Plataforma, 1 \\
		\addlinespace
		\stepcounter{requisitosNoFuncionales} RNF\therequisitosNoFuncionales &La plataforma cumplirá con el RGPD. Accesible por los profesionales autorizados. & Obligatorio & Características de la Plataforma, 8 \\
		\addlinespace
		\stepcounter{requisitosNoFuncionales}
		RNF\therequisitosNoFuncionales & Los recordatorios deben estar disponibles 24/7 para que los pacientes reciban la información a tiempo. & Obligatorio & Características de la Plataforma, 6 \\
		\addlinespace
		\stepcounter{requisitosNoFuncionales}
		RNF\therequisitosNoFuncionales &  Debe permitir la sincronización con calendarios personales, como Google Calendar o iCalendar & Opcional & Charla con Paciente (chatGPT) \\
		\addlinespace
		
		\bottomrule
	\end{tabular}
\end{table}

\section{Alcance del proyecto}
El proyecto abarca el diseño y desarrollo de una plataforma web de rehabilitación remota, accesible desde dispositivos móviles, tabletas y PCs. El sistema proporcionará a los pacientes y profesionales de la salud las herramientas necesarias para gestionar el proceso de rehabilitación de manera eficiente y personalizada. Los elementos clave son:


\begin{itemize}
	\item \textbf{Acceso Multidispositivo:} 
	Interfaz accesible y adaptada a diferentes dispositivos para pacientes y profesionales.

	
	\item \textbf{Gestión de Planes de Rehabilitación Personalizados:} 
	Creación, actualización y seguimiento de planes de rehabilitación personalizados por los profesionales de salud. Los pacientes podrán registrar su progreso diario y obtener incentivos a través de la gamificación.

	
	\item \textbf{Supervisión en Tiempo Real:} 
	 Integración con dispositivos \textit{wearables} para monitoreo del progreso del paciente y alertas automáticas para los profesionales.
	
	
	\item \textbf{Herramientas de Comunicación:} 
	Sistema de mensajería y videollamadas para interacción directa entre pacientes y profesionales, con programación de sesiones y recordatorios.
	
	
	\item \textbf{Seguridad y Privacidad:} 
	 Cumplimiento con el RGPD, garantizando la protección de datos y acceso seguro para profesionales autorizados.
	
	
	\item \textbf{Almacenamiento de Datos:} 
	 Almacenamiento seguro de registros de pacientes y generación de reportes para su seguimiento.
	
\end{itemize}

\subsection{Exclusiones}
\begin{itemize}
	\item No se desarrollará hardware propio (solo compatibilidad con dispositivos de terceros).
	\item No incluirá módulos de facturación o administración de clínicas.
	\item No ofrecerá diagnóstico médico automatizado.
\end{itemize}



\end{document}
