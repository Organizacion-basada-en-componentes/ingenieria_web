\documentclass{article}
\usepackage{fullpage}

%load needed packages
\usepackage{graphicx}
\usepackage{array}
\usepackage{booktabs}
\usepackage[utf8]{inputenc}
\usepackage[T1]{fontenc}
\usepackage{hyperref}

\usepackage[spanish]{babel} % Paquete para el idioma español
\usepackage{float}  % Necesario para [H]
\usepackage{listings}
\usepackage{xcolor}
\usepackage{longtable} 

\definecolor{codegreen}{HTML}{5AB2FF}
\definecolor{morado}{HTML}{AD88C6}
\definecolor{BG}{HTML}{EEEEEE}
\definecolor{azul}{HTML}{4D869C}
\definecolor{sqlblue}{HTML}{FF8C00} % Color para las palabras clave SQL

% Estilo para DDL
\lstdefinestyle{ddlstyle}{
	language=SQL,
	backgroundcolor=\color{BG},
	commentstyle=\color{codegreen},
	basicstyle=\ttfamily\small,
	keywordstyle=\color{azul},
	stringstyle=\color{morado},
	showstringspaces=false,
	breaklines=true,
	frame=shadowbox,
	numbers=left,
	numberstyle=\tiny\color{gray},
	captionpos=b,
}

% Estilo para SQL
\lstdefinestyle{sqlstyle}{
	language=SQL,
	backgroundcolor=\color{BG},
	commentstyle=\color{codegreen},
	basicstyle=\ttfamily\small,
	keywordstyle=\color{sqlblue}, % Color diferente para palabras clave SQL
	stringstyle=\color{morado},
	showstringspaces=false,
	breaklines=true,
	frame=shadowbox,
	numbers=left,
	numberstyle=\tiny\color{gray},
	captionpos=b,
}

\begin{document}
	
	% Portada
	\begin{titlepage}
		\centering
		\vspace*{3cm}
		
		% Título destacado
		{\Huge \textbf{Plataforma Web de Rehabilitación a Distancia}\\[0.5cm]}
		
		% Espacio y logotipo (si lo tienes, por ejemplo el logo de tu universidad)
		\vspace{2cm}
		\includegraphics[width=0.3\textwidth]{images/uma_logo.jpg}\\[1cm]
		
		% Nombre del autor
		{\LARGE \textbf{Organización basada en componentes}\\[0.5cm]}
		{\large \textit{Integrantes:}\\
			Cuevas Rodríguez, Marta\\
			de Pablo, Diego\\
			Silva Rodríguez, Alejandro\\
			Soriano Muñoz, Juan Ignacio\\
		}
		\vfill
		{\large \textit{Ingeniería web}\\
			Universidad de Málaga\\
		}
		
		\vfill
		
		% Fecha en la parte inferior de la página
		{\large Octubre 2024}
	\end{titlepage}
	
	% índice
	\tableofcontents
	
	\newpage
	
	\section{Introducción}

La rehabilitación es una fase crítica en el proceso de recuperación de pacientes que han sufrido lesiones, intervenciones quirúrgicas o padecen enfermedades crónicas. Tradicionalmente, la rehabilitación se realiza de manera presencial, lo que puede generar barreras logísticas, económicas y geográficas tanto para los pacientes como para los profesionales de la salud. Ante esta realidad, surge la necesidad de una Plataforma Web de Rehabilitación a Distancia, cuyo objetivo es facilitar el acceso a programas de rehabilitación personalizados, ofrecer seguimiento remoto y mejorar la calidad de vida de los pacientes sin la necesidad de visitas constantes a centros de rehabilitación.
\\
\\
Este proyecto plantea el desarrollo de una plataforma web integral que permita a los pacientes recibir tratamientos de rehabilitación de manera remota, mientras que los profesionales de la salud pueden monitorizar el progreso y ajustar las terapias en tiempo real. Los principales stakeholders involucrados en este proyecto incluyen a pacientes, profesionales de la salud (fisioterapeutas, médicos rehabilitadores) y desarrolladores de software. Los pacientes se beneficiarán de un acceso más flexible a sus tratamientos, mientras que los profesionales podrán optimizar el seguimiento clínico y ajustar terapias de manera eficiente.
\\
\\
Entre los posibles casos de uso se encuentran situaciones como la rehabilitación de un paciente con una lesión muscular, que puede realizar sus ejercicios desde casa bajo la supervisión de un fisioterapeuta a través de videollamadas, o un paciente crónico que, mediante dispositivos de telemetría y un registro de ejercicios, permite que su progreso sea monitorizado de forma continua.


	
	\section{Descripción de Stakeholders}
	
	Un stakeholder (o parte interesada) es cualquier persona, grupo u organización que tiene un interés o impacto directo en un proyecto, producto o empresa, o que se ve afectada por los resultados de dicho proyecto. 
	
	\begin{table}[ht]
		\centering
		\caption{Stakeholders Principales}
		\begin{tabular}{@{} p{3cm} p{6cm} p{7cm} @{}}
			\toprule
			\textbf{Nombre} & \textbf{Representa} & \textbf{Rol} \\
			\midrule
			Profesional médico & 
			Especialista de la salud encargado del seguimiento y control de la rehabilitación a distancia & 
			Supervisa y evalúa el progreso de los pacientes. \newline Personaliza los planes de tratamiento, proporciona feedback y ajusta terapias en tiempo real. \\
			
			\addlinespace
			
			Paciente & 
			Usuarios finales que necesitan realizar terapias de rehabilitación a distancia. & 
			Participan activamente usando la plataforma para seguir sus planes de rehabilitación, registrar su progreso y recibir retroalimentación. Su satisfacción es esencial, ya que su experiencia definirá el éxito del proyecto. \\
			
			\addlinespace
			
			Desarrollador  Software & 
			Equipo encargado del desarrollo del software de la aplicación o página web & 
			Desarrolla y mantiene la funcionalidad del sistema, asegurándose de que los datos de los pacientes se almacenen y se procesen correctamente. \\
			\bottomrule
		\end{tabular}
	\end{table}
	
	

	
	
	\section{Requisitos}
	
	\subsection{Funcionales}
	
\newcounter{requisitosFuncionales}
\begin{longtable}{@{} p{2.5cm} p{6.5cm} p{3cm} p{3cm} @{}}
	\caption{Requisitos Funcionales}\\
	\toprule
	\textbf{ID} & \textbf{Descripción} & \textbf{Obligatoriedad} & \textbf{Trazabilidad} \\
	\midrule
	\endfirsthead
	
	\toprule
	\textbf{ID} & \textbf{Descripción} & \textbf{Obligatoriedad} & \textbf{Trazabilidad} \\
	\midrule
	\endhead
	
	\addlinespace 
	RF\therequisitosFuncionales & Los usuarios podrán registrarse en la plataforma según su rol. & Obligatorio & - \\
	\addlinespace \stepcounter{requisitosFuncionales}
	RF\therequisitosFuncionales & Los usuarios podrán iniciar sesión en la plataforma, accediendo a distintas partes del sistema dependiendo su rol. & Obligatorio & - \\  
	\addlinespace \stepcounter{requisitosFuncionales}
	RF\therequisitosFuncionales & Los pacientes podrán acceder a su plan de rehabilitación desde su perfil. & Obligatorio & RF1\\ 
	\addlinespace \stepcounter{requisitosFuncionales}
	RF\therequisitosFuncionales & Los pacientes podrán visualizar su historial de ejercicios desde su perfil. & Obligatorio & RF2 \\ 
	\addlinespace \stepcounter{requisitosFuncionales}
	RF\therequisitosFuncionales & Los pacientes podrán reportar su progreso desde su perfil. & Obligatorio & RF2 \\ 
	\addlinespace \stepcounter{requisitosFuncionales}
	RF\therequisitosFuncionales & Los pacientes recibirán recordatorios de las sesiones desde su perfil. & Opcional & RF2 \\
	\addlinespace \stepcounter{requisitosFuncionales}
	RF\therequisitosFuncionales & Los pacientes podrán registrar sus resultados y cumplimiento diario del plan de ejercicios, especificando series, repeticiones y niveles de dolor. & Obligatorio & - \\ 
	RF\therequisitosFuncionales & Los pacientes podrán agendar consultas a través de videollamadas con su fisioterapeuta. & Obligatorio & - \\
	\addlinespace \stepcounter{requisitosFuncionales}
	RF\therequisitosFuncionales & Los profesionales podrán acceder a la información de sus pacientes. & Obligatorio & RF1 \\ 
	\addlinespace \stepcounter{requisitosFuncionales}
	RF\therequisitosFuncionales & Los profesionales podrán programar sesiones de seguimiento con sus pacientes. & Obligatorio & RF5 \\ 
	\addlinespace \stepcounter{requisitosFuncionales}
	RF\therequisitosFuncionales & Los profesionales podrán actualizar los planes de rehabilitación de los pacientes. & Obligatorio & RF2 \\ 
	\addlinespace \stepcounter{requisitosFuncionales}
	RF\therequisitosFuncionales & Los profesionales podrán revisar los informes generados de los pacientes. & Obligatorio & RF7, RF4\\ 
	\addlinespace \stepcounter{requisitosFuncionales}
	 

	RF\therequisitosFuncionales & Los profesionales podrán crear planes de rehabilitación personalizados, seleccionando una serie de ejercicios recomendados en función de la condición médica del paciente. & Obligatorio & - \\ 
	\addlinespace \stepcounter{requisitosFuncionales}

	RF\therequisitosFuncionales & La plataforma proporcionará una base de datos con plantillas de ejercicios clasificados por tipos de lesiones o enfermedades. & Obligatorio & - \\ 
	\addlinespace \stepcounter{requisitosFuncionales}

	RF\therequisitosFuncionales & Los profesionales podrán modificar y ajustar los planes de rehabilitación según el progreso observado en cada paciente. & Obligatorio & RF11 \\ 
	\addlinespace \stepcounter{requisitosFuncionales}
	RF\therequisitosFuncionales & La plataforma recopilará datos de dispositivos wearables como frecuencia cardíaca, calorías quemadas y tiempo de actividad. & Obligatorio & - \\
	\addlinespace \stepcounter{requisitosFuncionales}
	RF\therequisitosFuncionales & Si se detecta un patrón inusual (dolor intenso o incumplimiento de ejercicio), se enviará una alerta al profesional asignado. & Obligatorio & R14 \\
	\addlinespace \stepcounter{requisitosFuncionales}
	RF\therequisitosFuncionales & El sistema de mensajería permitirá la comunicación directa entre paciente y profesional. & Obligatorio & R6 \\
	\addlinespace \stepcounter{requisitosFuncionales}
	RF\therequisitosFuncionales & La plataforma generará informes automáticos sobre el progreso del paciente, incluyendo métricas clave como el cumplimiento del tratamiento, el dolor reportado y la movilidad mejorada. & Obligatorio & R4 \\
	\addlinespace \stepcounter{requisitosFuncionales}
	RF\therequisitosFuncionales & Los profesionales podrán acceder a paneles interactivos que proporcionan una visión general del progreso del paciente, permitiendo ajustes rápidos en el tratamiento si es necesario. & Obligatorio & R18 \\
	\bottomrule
\end{longtable}



\subsection{No funcionales}

\newcounter{requisitosNoFuncionales}
\begin{table}[H]
	\centering
	\caption{Requisitos No Funcionales}
	\begin{tabular}{@{} p{2.5cm} p{6.5cm} p{3cm} @{}}
		\toprule
		\textbf{ID} & \textbf{Descripción} & \textbf{Obligatoriedad} \\
		\midrule
		\addlinespace
		\stepcounter{requisitosNoFuncionales} RNF\therequisitosNoFuncionales & La plataforma deberá ser accesible desde dispositivos móviles, tabletas y PC. & Obligatorio \\
		\addlinespace
		\stepcounter{requisitosNoFuncionales} RNF\therequisitosNoFuncionales & Se debe implementar un sistema de autenticación segura para proteger la privacidad y seguridad de los datos personales y médicos de los usuarios. & Obligatorio\\
			\stepcounter{requisitosNoFuncionales} RNF\therequisitosNoFuncionales & Todos los datos personales y médicos estarán protegidos mediante cifrado. & Obligatorio \\
		\addlinespace
		\stepcounter{requisitosNoFuncionales} RNF\therequisitosNoFuncionales &La plataforma cumplirá con el RGPD y otras normativas relacionadas con la protección de datos. Accesible por los profesionales autorizados. & Obligatorio \\
		\addlinespace
		\stepcounter{requisitosNoFuncionales}
		RNF\therequisitosNoFuncionales & La funcionalidad de recordatorios y notificaciones debe estar disponible las 24 horas para garantizar que los pacientes reciban la información a tiempo. & Obligatorio \\
		\addlinespace
		\stepcounter{requisitosNoFuncionales}
		RNF\therequisitosNoFuncionales &  Debe permitir la sincronización con calendarios personales de los pacientes, como Google Calendar o iCal. & Opcional \\
		\addlinespace
		\bottomrule
	\end{tabular}
\end{table}
	

\end{document}
